In this experiment, we would like to understand how data from a particle accelerator is analysed. For this, we consider the data from OPAL (Omni-Purpose Apparatus at LEP) experiment carried out at the Large electron-positron collider (LEP). In this particular report, we try to deduce different properties of the $Z^0$ boson. For this, we first carry out event display analysis on a smaller set of data to understand how we can separate out different channels corresponding to this process. And after this, we carry out statistical analysis on the real world data to deduce important physical quantities, like the mass and decay width of $Z^0$ and partial decay widths.\\
This report is divided into multiple sections. The first section discusses the requisite theoretical knowledge to understand this report. This is then followed by the section which discusses the solutions to pre-lab exercises, which are essential in understanding both the theoretical and experimental concepts. This is followed by analysis, where we explore and understand the data using various methods and finally discuss various physical quantities. This is then finally followed by conclusion, which summarises what had been done. The final parts of the report contain supplementary data and plots. All the data and source code for the analysis can be found \href{https://github.com/Mrunmoy-Jena/E213Lab}{here}\footnote{https://github.com/Mrunmoy-Jena/E213Lab}.