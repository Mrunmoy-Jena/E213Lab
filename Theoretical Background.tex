\section{The Standard Model- A Brief Overview}
The Standard Model of particle physics which has been one of the most successful and well-tested theories so far, provides the most fundamental description of nature by incorporating the elementary particles and their interactions. These elementary particles are categorized into two families: fermions (having half integer spins) which form matter as we know it, and bosons (with integer spins) serving as mediators of the three fundamental forces. While electromagnetic interactions are mediated by the photon ($\gamma$), strong and weak interactions are mediated by gluons (g) and by $W^{\pm}, Z^{0}$ bosons respectively. A fundamental particle that mediates gravitation has been only postulated theoretically, and is left out of the Standard Model, since the effects of gravity are too weak to play any important role in the realm of particle physics. The fermions consists of three generations of quarks and leptons. The quarks have six flavours: up (u), down (d), charm (c), strange (s), top (t) and bottom (b). Similarly, the leptons consist of the electron ($e$), muon ($\mu$) and tau ($\tau$), each having its own associated charge-less and almost massless neutrino ($\nu_{e}$, $\nu_{\mu}$ and $\nu_{\tau}$). Furthermore, each particle in the standard model has its own antiparticle. The quarks are able to form composite particles in either three quark combinations, called baryons ($qqq$/$\bar{q}\bar{q}\bar{q}$) or a quark-antiquark pair, called a meson ($q\bar{q}$). Mathematically, the elementary particles are described as elements of representations of certain symmetry groups. The gauge fields that couple to these particles (i.e. mediate the interactions) arise naturally as a consequence of invariance of their corresponding Lagrangian under local group transformations \cite{thomson_2013}. As such, the gauge symmetry that governs the Standard Model is given by: $$SU(3)_{\mathrm{Colour}}\times SU(2)_{\mathrm{Left\ chiral}}\times U(1)_{\mathrm{Y}(\mathrm{Weak \ hypercharge})}$$

\section{The Unified Theory of Electromagnetic and Weak Interactions}
Since the experiment deals with verifying some of the properties of the $Z^{0}$ bosons, it is of interest to touch upon the theory of electroweak unification.

While electromagnetism and the theory of weak interactions were formulated separately, it was later on postulated that at higher energies ($\sim$ 246 GeV \cite{enwiki:1085396801}), both these interactions would be unified into a single force. As such, the GSW(Glashow-Salam-Weinberg) electroweak model was developed in the 1960s to describe this unified force. 

One finds that imposing the principle of local gauge invariance on the $SU(2)_{L}$ symmetry group leads to the introduction of three gauge fields: $W^{(1)},\ W^{(2)}$ and $W^{(3)}$ (or $W^{0}$ in some references) \cite{thomson_2013}. The physical $W^{+}$ and $W^{-}$ bosons (that mediate the weak charged current interaction) can be seen as the linear combinations: 
\begin{equation}
W^{\pm}=\dfrac{1}{\sqrt{2}}\left(W^{(1)}\mp W^{(2)}\right)
\end{equation}
However, the $W^{(3)}$ field has no physical interpretation of its own. Therefore an additional symmetry, the $U(1)_{Y}$ group is introduced. The field $B$ (or equivalently $Y^{0}$) arising as a consequence of this new symmetry, similarly does not have a physical meaning on its own. Rather, it was seen that linear combinations of the $W^{(3)}$ ($W^{0}$) and $B$ ($Y^{0}$) fields gives rise to the  photon and the $Z^{0}$ boson:

\begin{equation}
\begin{pmatrix} 
\gamma \\ 
Z^{0} 
\end{pmatrix}
= 
\begin{pmatrix}
\cos \theta_{W} & \sin \theta_{W} \\
-\sin \theta_{W} & \cos \theta_{W} 
\end{pmatrix}
\begin{pmatrix}
B \\
W^{(3)}
\end{pmatrix}
\end{equation}

In addition to this, it is to be noted that the gauge fields $W^{(1),(2),(3)}$ and $B$ have to be massless, in order to respect gauge invariance under local $SU(2)_{L}\times U(1)_{Y}$ gauge transformation. However, the physical gauge bosons $W^{\pm}$ and $Z^{0}$ are predicted to be massive, whereas the photon should remain massless. To explain this, the concept of electroweak spontaneous symmetry breaking was introduced. A massive scalar field (the Higgs field) is introduced, to which these bosons ($W^{\pm}$, $Z^{0}$) must couple to, in order to get their physical masses, while the photon does not interact with it \cite{Dooling:207610}. The intricacies of the Higgs mechanism are not of immediate interest here, and can be understood from standard references \cite{thomson_2013, Griffiths:111880}.
