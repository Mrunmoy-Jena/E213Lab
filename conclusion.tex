Our objective to understand how analysis is carried out with data from a particle collider was achieved in this experiment. We started with event display analysis, which let us develop a cut criteria to separate out different channels involved in the decay of $Z^0$. This was then further refined with the help of \textit{Monte-Carlo} simulation events. Once we had the final cuts, we went about calculating the efficiency matrix, which was used to determine the actual counts. This was then used to calculate the cross sections. The cross section values at $Z^0$ resonance was then used to verify lepton universality. We noted that even though the cross sections are about the same for the three channels, they are several standard deviations away from the literature value. The measurement of forward-backward asymmetry in the muon channel helped us determine the weak mixing angle. The literature value lies within $1\sigma$ of the value obtained here. We then fit the cross sections of different channels at different centre of mass energies to \textit{Breit-Wigner curve}, which enabled us to determine the mass and decay width of $Z^0$ and the partial decay widths of the different channels. The literature value for the mass of $Z^0$ lies within $4\sigma$ of the calculated value and the literature value for the decay width of $Z^0$ lies within $2\sigma$ of the calculated value, which is impressive. The literature value for partial decay widths of all the channels lie within $1\sigma$ of the calculated values. We also determined the maximum possible light neutrino generations and eliminated the possibility of a fourth light neutrino within $1\sigma$. All this analysis ultimately relied on the measurement of $N$, which implies that we could improve both the accuracy and precision of the results by significantly increasing $N$. Systematic errors, if known quantitatively, would also improve our results.